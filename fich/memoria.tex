\documentclass[11pt,a4paper]{article}
\usepackage{ucs}
\usepackage[utf8x]{inputenc}
\usepackage[spanish]{babel}
\usepackage{longtable}
\usepackage{lscape}
\usepackage{anysize}
\begin{document}
% CUERPO
\marginsize{3cm}{2cm}{2.5cm}{2.5cm}
\begin{center}
{\huge \textbf {MEMORIA DE PYMES}}
\end{center}

\begin{center}
{\Large \textbf {EJERCICIO (-EJERCICIO-)}}
\end{center}

Que, relativa al ejercicio (-EJERCICIO-) cerrado al (-CIERRE-), somete el Administrador de la entidad (-EMPRESA-)., a la Junta General Ordinaria de la Sociedad que ha de sancionar las Cuentas Anuales correspondientes a dicho ejercicio, el cual empieza el (-APERTURA-) y termina el (-CIERRE-).

\section{ACTIVIDAD DE LA EMPRESA}

La Compañía (-EMPRESA-) se constituye el (-FECHACONST-) en xxxxxxx. Su domicilio social se encuentra en (-CIUDAD-), provincia de (-PROVINCIA-), (-DOMICILIO-). Su objeto social es (-OBJETO-) y se dedica a (-ACTIVIDAD-)


\section{Bases de presentación de las cuentas anuales}

  \subsection{Imagen fiel}
Las cuentas anuales de (-EMPRESA-) reflejan la imagen fiel del patrimonio, de la situación financiera y de los resultados de la empresa.

  \subsection{Principios contables no obligatorios aplicados}
No se han aplicado principios contables no obligatorios.

  \subsection{Aspectos críticos de la valoración y estimación de la incertidumbre}
En el presente ejercicio no existen cambios en estimaciones contables que sean significativos y que afecten al ejercicio actual o que se espera que pueda afectar a los ejercicios futuros.

  \subsection{Comparación de la información}
La disposición transitoria cuarta permite que las cuentas anuales correspondientes al ejercicio que se inicie a partir de la entrada en vigor del Plan General de Contabilidad de 2007, se considerarán cuentas anuales iniciales, por lo que no se reflejarán cifras comparativas en las referidas cuentas, aunque en la memoria de dichas cuentas anuales iniciales se reflejarán las cifras del activo, pasivo y cuenta de resultados del ejercicio anterior.

Sin perjuicio de lo indicado en los apartados siguientes respecto a los cambios en criterios contables y corrección de errores, en este apartado se incorporará la siguiente información:

\begin{enumerate}
 \item Razones excepcionales que justifican la modificación de la estructura del balance, de la cuenta de pérdidas y ganancias, del estado de cambios en el patrimonio neto y, en caso de confeccionarse, del estado de flujos de efectivo del ejercicio anterior.
 \item Explicación de las causas que impiden la comparación de las cuentas anuales del ejercicio con las del precedente.
 \item Explicación de la adaptación de los importes del ejercicio precedente para facilitar la comparación y, en caso contrario, las razones excepcionales que han hecho impracticable la reexpresión de las cifras comparativas.
\end{enumerate}

  \subsection{Elementos recogidos en varias partidas}
Identificación de los elementos patrimoniales, con su importe, que estén registrados en dos o más partidas del balance, con indicación de éstas y del importe incluido en cada una de ellas.
No hay elementos patrimoniales que estén registrados en dos o más partidas del balance.

  \subsection{Cambios en criterios contables}
Explicación detallada de los ajustes por cambios en criterios contables realizados en el ejercicio, señalándose las razones por las cuales el cambio permite una información más fiable y relevante.

Si la aplicación retroactiva fuera impracticable, se informará sobre tal hecho, las circunstancias que lo explican y desde cuándo se ha aplicado el cambio en el criterio contable.

No hay ajustes por cambios en criterios contables realizados en el ejercicio.

\subsection{Corrección de errores}
Explicación detallada de los ajustes por corrección de errores realizados en el ejercicio, indicándose la naturaleza del error.

Si la aplicación retroactiva fuera impracticable, se informará sobre tal hecho, las circunstancias que lo explican y desde cuándo se ha corregido el error.

No se conocen errores realizados en el ejercicio; por tanto, no se ha realizado ajustes por correcciones para tal fin.

\section{Aplicación de resultados}
La propuesta de aplicación del resultado del ejercicio se realiza de acuerdo con el siguiente esquema:
\begin{center}
% use packages: array
\begin{tabular}{|l|r|}
\hline
\textbf {BASE DE REPARTO} & \textbf {IMPORTE} \\ 
\hline
Saldo de la cuenta de pérdidas y ganancias & 100,00 \\ 
\hline
Remanente & 10,00 \\ 
\hline
Reservas voluntarias & 15,00 \\ 
\hline
Otras reservas de libre disposición & 5,00 \\ 
\hline
\textbf {TOTAL} & \textbf {130,00} \\
\hline
\end{tabular}
\end{center}


\begin{center}
% use packages: array
\begin{tabular}{|l|r|}
\hline
\textbf {APLICACIÓN} & \textbf {IMPORTE} \\ 
\hline
A reserva legal & 100,00 \\ 
\hline
A reservas especiales & 10,00 \\ 
\hline
A reservas voluntarias & 15,00 \\ 
\hline
A ............................... & 5,00 \\ 
\hline
A dividendos & 5,00 \\ 
\hline
A ............................... & 5,00 \\ 
\hline
A compensación de pérdidas de ejercicios anteriores & 5,00 \\ 
\hline
\textbf {TOTAL} & \textbf {130,00} \\
\hline
\end{tabular}
\end{center}

En el caso de distribución de dividendos a cuenta en el ejercicio, se deberá indicar el importe de los mismos e incorporar el estado contable previsional formulado preceptivamente para poner de manifiesto la existencia de liquidez suficiente. Dicho estado contable deberá abarcar un período de un año desde que se acuerde la distribución del dividendo a cuenta.

Limitaciones para la distribución de dividendos.

La distribución prevista en el reparto de resultados del ejercicio cumple con los requisitos y limitaciones establecidas en los estatutos de la Sociedad y en la normativa legal.


\section{Normas de registro y valoración}
\begin{enumerate}
 \item Inmovilizado intangible; indicando los criterios utilizados de capitalización o activación, amortización y correcciones valorativas por deterioro.\\
Deberá indicarse de forma detallada el criterio de valoración seguido para calcular el valor recuperable de los inmovilizados intangibles con vida útil indefinida.
\item Inmovilizado material; indicando los criterios sobre amortización, correcciones valorativas por deterioro y reversión de las mismas, capitalización de gastos financieros, costes de ampliación, modernización y mejoras, costes de desmantelamiento o retiro, así como los costes de rehabilitación del lugar donde se asiente un activo y los criterios sobre la determinación del coste de los trabajos efectuados por la empresa para su inmovilizado.\\
Además se precisarán los criterios de contabilización de contratos de arrendamiento financiero y otras operaciones de naturaleza similar.
\item Se señalará el criterio para calificar los terrenos y construcciones como inversiones inmobiliarias, especificando para éstas los criterios señalados en el apartado anterior.\\
Además se precisarán los criterios de contabilización de contratos de arrendamiento financiero y otras operaciones de naturaleza similar.
\item Permutas; indicando el criterio seguido y la justificación de su aplicación, en particular, las circunstancias que han llevado a calificar una permuta de carácter comercial.
\item Activos financieros y pasivos financieros; se indicará:
   \begin{enumerate}
    \item Criterios empleados para la calificación y valoración de las diferentes categorías de activos financieros y pasivos financieros, así como para el reconocimiento de cambios de valor razonable; en particular, las razones por las que los valores emitidos por la empresa que, de acuerdo con el instrumento jurídico empleado, en principio debieran haberse clasificado como instrumentos de patrimonio, han sido contabilizados como pasivos financieros.

    \item Los criterios aplicados para determinar la existencia de evidencia objetiva de deterioro, así como el registro de la corrección de valor y su reversión y la baja definitiva de activos financieros deteriorados. En particular, se destacarán los criterios utilizados para calcular las correcciones valorativas relativas a los deudores comerciales y otras cuentas a cobrar. Asimismo, se indicarán los criterios contables aplicados a los activos financieros cuyas condiciones hayan sido renegociadas y que, de otro modo, estarían vencidos o deteriorados.

    \item Criterios empleados para el registro de la baja de activos financieros y pasivos financieros.

    \item Inversiones en empresas del grupo, multigrupo y asociadas; se informará sobre el criterio seguido en la valoración de estas inversiones, así como el aplicado para registrar las correcciones valorativas por deterioro.

    \item Los criterios empleados en la determinación de los ingresos o gastos procedentes de las distintas categorías de activos y pasivos financieros: intereses, primas o descuentos, dividendos, etc.

   \end{enumerate}

\item Valores de capital propio en poder de la empresa; indicando los criterios de valoración y registro empleados.

\item Existencias; indicando los criterios de valoración y, en particular, precisando los seguidos sobre correcciones valorativas por deterioro y capitalización de gastos financieros.

\item Transacciones en moneda extranjera; indicando los criterios de valoración de las transacciones en moneda extranjera y criterios de imputación de las diferencias de cambio.

\item Impuestos sobre beneficios; indicando los criterios utilizados para el registro y valoración de activos y pasivos por impuesto diferido.

\item Ingresos y gastos; indicando los criterios generales aplicados. En particular, en relación con las prestaciones de servicios realizadas por la empresa se indicarán los criterios utilizados para la determinación de los ingresos; en concreto, se señalarán los métodos empleados para determinar el porcentaje de realización en la prestación de servicios y se informará en caso de que su aplicación hubiera sido impracticable.

\item Provisiones y contingencias; indicando el criterio de valoración, así como, en su caso, el tratamiento de las compensaciones a recibir de un tercero en el momento de liquidar la obligación. En particular, en relación con las provisiones deberá realizarse una descripción general del método de estimación y cálculo de cada uno de los riesgos.

\item Subvenciones, donaciones y legados; indicando el criterio empleado para su clasificación y, en su caso, su imputación a resultados.

\item Negocios conjuntos; indicando los criterios seguidos por la empresa para integrar en sus cuentas anuales los saldos correspondientes al negocio conjunto en que participe.

\item Criterios empleados en transacciones entre partes vinculadas.

\end{enumerate}

\section{Inmovilizado material, intangible e inversiones inmobiliarias}
(ver variación en caso microempresas)
Análisis del movimiento durante el ejercicio de cada uno de estos epígrafes del balance y de sus correspondientes amortizaciones acumuladas y correcciones valorativas por deterioro de valor acumuladas; indicando lo siguiente:

\begin{center}
% use packages: array
\begin{tabular}{|l|r|r|r|r|}
\hline
\textbf {Descripción} & \textbf {Saldo inicial} & \textbf {Entradas} & \textbf {Salidas} & \textbf {Saldo final} \\ 
\hline
Inmovilizado xxx & 100,00 & 10,00 &  & 110,00 \\
\hline
Inmovilizado yyy & 200,00 &  & 20,00 & 120,00 \\
\hline
\end{tabular}
\end{center}


En particular, se detallarán los inmovilizados intangibles con vida útil indefinida y las razones sobre las que se apoya la estimación de dicha vida útil indefinida.

También se especificará la información relativa a inversiones inmobiliarias, incluyéndose además una descripción de las mismas.

Si hubiera algún epígrafe significativo, por su naturaleza o por su importe, se facilitará la pertinente información adicional.

Arrendamientos financieros y otras operaciones de naturaleza similar sobre activos no corrientes. En particular, precisando de acuerdo con las condiciones del contrato: coste del bien en origen, duración del contrato, años transcurridos, cuotas satisfechas en años anteriores y en el ejercicio, cuotas pendientes y, en su caso, valor de la opción de compra.

\section{Activos financieros}
\begin{enumerate}
\item Se revelará el valor en libros de cada una de las categorías de activos financieros señaladas en la norma de registro y valoración novena, salvo las inversiones en el patrimonio de empresas del grupo, multigrupo y asociadas.
A estos efectos se desglosará cada epígrafe atendiendo a las categorías establecidas en la norma de registro y valoración novena. Se deberá informar sobre las clases definidas por la empresa (CUADRO)

\item Se presentará para cada clase de activos financieros, un análisis del movimiento de las cuentas correctoras representativas de las pérdidas por deterioro originadas por el riesgo de crédito.

\item Cuando los activos financieros se hayan valorado por su valor razonable, se indicará:
  \begin{enumerate}
    \item Si el valor razonable se determina, en su totalidad o en parte, tomando como referencia los precios cotizados en mercados activos o se estima utilizando modelos y técnicas de valoración. En este último caso, se señalarán los principales supuestos en que se basan los citados modelos y técnicas de valoración.

    \item Por categoría de activos financieros, el valor razonable y las variaciones en el valor registradas en la cuenta de pérdidas y ganancias.

    \item Con respecto a los instrumentos financieros derivados, se informará sobre la naturaleza de los instrumentos y las condiciones importantes que puedan afectar al importe, al calendario y a la certidumbre de los futuros flujos de efectivo.
  \end{enumerate}

\item Empresas del grupo, multigrupo y asociadas.

Se detallará información sobre las empresas del grupo, multigrupo y asociadas, incluyendo:
  \begin{enumerate}
   \item Denominación, domicilio y forma jurídica de las empresas del grupo, especificando para cada una de ellas:
     \begin{itemize}
       \item Actividades que ejercen.
       \item Fracción de capital y de los derechos de voto que se posee directa e indirectamente, distinguiendo entre ambos.
       \item Importe del capital, reservas, otras partidas del patrimonio neto y resultado del último ejercicio, diferenciando el resultado de explotación.
       \item Valor según libros de la participación en capital.
       \item Dividendos recibidos en el ejercicio.
       \item Indicación de si las acciones cotizan o no en Bolsa y, en su caso, cotización media del último trimestre del ejercicio y cotización al cierre del ejercicio.
     \end{itemize}

   \item La misma información que la del punto anterior respecto de las empresas multigrupo, asociadas, aquellas en las que aun poseyendo más del 20\% del capital la empresa no se ejerza influencia significativa y aquellas en las que la sociedad sea socio colectivo. Asimismo, se informará sobre las contingencias en las que se haya incurrido en relación con dichas empresas. Si la empresa ejerce influencia significativa sobre otra poseyendo un porcentaje inferior al 20\% del capital o si poseyendo más del 20\% del capital no se ejerce influencia significativa, se explicarán las circunstancias que afectan a dichas relaciones.

   \item Se detallarán las adquisiciones realizadas durante el ejercicio que hayan llevado a calificar a una empresa como dependiente, indicándose la fracción de capital y el porcentaje de derechos de voto adquiridos.

   \item Notificaciones efectuadas, en cumplimiento de lo dispuesto en el artículo 86 del Texto Refundido de la Ley de Sociedades Anónimas, a las sociedades participadas, directa o indirectamente, en más de un 10%

   \item Importe de las correcciones valorativas por deterioro registradas en las distintas participaciones, diferenciando las reconocidas en el ejercicio de las acumuladas.
  \end{enumerate}

\end{enumerate}



\begin{landscape}
\begin{center}
% use packages: array
\begin{tabular}{|p{3cm}|r|r|r|r|r|r|r|r|r|r|r|r|r|r|}
\hline
 & \multicolumn{14}{|c|} {\small{CLASES}} \\
\hline
 & \multicolumn{6}{|c|} {\tiny{Instrumentos financieros a largo plazo}}
 & \multicolumn{6}{|c|} {\tiny{Instrumentos financieros a corto plazo}}
 & \multicolumn{2}{|c|} {} \\
\hline
 & \multicolumn{2}{|p{2cm}|} {\tiny{Instrumentos de patrimonio}}
 & \multicolumn{2}{|p{2cm}|} {\tiny{Valores representativos de deuda}}
& \multicolumn{2}{|p{2cm}|} {\tiny{Créditos Derivados Otros}}
& \multicolumn{2}{|p{2cm}|} {\tiny{Instrumentos de patrimonio}}
& \multicolumn{2}{|p{2cm}|} {\tiny{Valores representativos de deuda}}
& \multicolumn{2}{|p{2cm}|} {\tiny{Créditos Derivados Otros}}
& \multicolumn{2}{|p{2cm}|} {\tiny{Total}} \\
\hline
\small{CATEGORÍAS} & \tiny{Ej.x} & \tiny{Ej.x-1} & \tiny{Ej.x} & \tiny{Ej.x-1} & \tiny{Ej.x} & \tiny{Ej.x-1} & \tiny{Ej.x} & \tiny{Ej.x-1} & \tiny{Ej.x} & \tiny{Ej.x-1} & \tiny{Ej.x} & \tiny{Ej.x-1} & \tiny{Ej.x} & \tiny{Ej.x-1} \\
\hline
\tiny{Activos a valor razonable con cambios en pérdidas y ganancias} &  &  &  &  &  &  &  &  &  &  &  &  &  &  \\ 
\hline
\tiny{Inversiones mantenidas hasta el vencimiento} &  &  &  &  &  &  &  &  &  &  &  &  &  &  \\ 
\hline
\tiny{Préstamos y partidas a cobrar} &  &  &  &  &  &  &  &  &  &  &  &  &  &  \\ 
\hline
\tiny{Activos disponibles para la venta} &  &  &  &  &  &  &  &  &  &  &  &  &  &  \\ 
\hline
\tiny{Derivados de cobertura} &  &  &  &  &  &  &  &  &  &  &  &  &  &  \\ 
\hline
\tiny{TOTAL} &  &  &  &  &  &  &  &  &  &  &  &  &  &  \\ 
\hline
\end{tabular}
\end{center}
\end{landscape}


\section{Pasivos financieros}
\begin{enumerate}
\item Se revelará el valor en libros de cada una de las categorías de pasivos financieros señalados en la norma de registro y valoración novena.
A estos efectos se desglosará cada epígrafe atendiendo a las categorías establecidas en la norma de registro y valoración novena. Se deberá informar sobre las clases definidas por la empresa.
(ver cuadro)

\item Información sobre:
   \begin{enumerate}
    \item El importe de las deudas que venzan en cada uno de los cinco años siguientes al cierre del ejercicio y del resto hasta su último vencimiento. Estas indicaciones figurarán separadamente para cada uno de los epígrafes y partidas relativos a deudas, conforme al modelo de balance.
    \item El importe de las deudas con garantía real, con indicación de su forma y naturaleza.
   \end{enumerate}

\item En relación con los préstamos pendientes de pago al cierre del ejercicio, se informará de:
 \begin{itemize}
 \item Los detalles de cualquier impago del principal o intereses que se haya producido durante el ejercicio.
 \item El valor en libros en la fecha de cierre del ejercicio de aquellos préstamos en los que se hubiese producido un incumplimiento por impago, y
 \item Si el impago ha sido subsanado o se han renegociado las condiciones del préstamo, antes de la fecha de formulación de las cuentas anuales.
 \end{itemize}
\end{enumerate}

\begin{landscape}
\begin{center}
% use packages: array
\begin{tabular}{|p{3cm}|r|r|r|r|r|r|r|r|r|r|r|r|r|r|}
\hline
 & \multicolumn{14}{|c|} {\small{CLASES}} \\
\hline
 & \multicolumn{6}{|c|} {\tiny{Instrumentos financieros a largo plazo}}
 & \multicolumn{6}{|c|} {\tiny{Instrumentos financieros a corto plazo}}
 & \multicolumn{2}{|c|} {} \\
\hline
 & \multicolumn{2}{|p{2cm}|} {\tiny{Deudas con entidades de crédito}}
 & \multicolumn{2}{|p{2cm}|} {\tiny{Obligaciones y otros valores negociables}}
& \multicolumn{2}{|p{2cm}|} {\tiny{Derivados Otros}}
& \multicolumn{2}{|p{2cm}|} {\tiny{Deudas con entidades de crédito}}
& \multicolumn{2}{|p{2cm}|} {\tiny{Obligaciones y otros valores negociables}}
& \multicolumn{2}{|p{2cm}|} {\tiny{Derivados Otros}}
& \multicolumn{2}{|p{2cm}|} {\tiny{Total}} \\
\hline
\small{CATEGORÍAS} & \tiny{Ej.x} & \tiny{Ej.x-1} & \tiny{Ej.x} & \tiny{Ej.x-1} & \tiny{Ej.x} & \tiny{Ej.x-1} & \tiny{Ej.x} & \tiny{Ej.x-1} & \tiny{Ej.x} & \tiny{Ej.x-1} & \tiny{Ej.x} & \tiny{Ej.x-1} & \tiny{Ej.x} & \tiny{Ej.x-1} \\
\hline
\tiny{Débitos y partidas a pagar} &  &  &  &  &  &  &  &  &  &  &  &  &  &  \\ 
\hline
\tiny{Pasivos a valor razonable con cambios en pérdidas y ganancias} &  &  &  &  &  &  &  &  &  &  &  &  &  &  \\ 
\hline
\tiny{Otros} &  &  &  &  &  &  &  &  &  &  &  &  &  &  \\ 
\hline
\tiny{TOTAL} &  &  &  &  &  &  &  &  &  &  &  &  &  &  \\ 
\hline
\end{tabular}
\end{center}
\end{landscape}


\section{Fondos propios}
\begin{enumerate}
 \item Cuando existan varias clases de acciones o participaciones en el capital, se indicará el número y el valor nominal de cada una de ellas, distinguiendo por clases, así como los derechos otorgados a las mismas y las restricciones que puedan tener. También, en su caso, se indicará para cada clase los desembolsos pendientes, así como la fecha de exigibilidad.

\item Circunstancias específicas que restringen la disponibilidad de las reservas.

\item Número, valor nominal y precio medio de adquisición de las acciones o participaciones propias en poder de la sociedad o de un tercero que obre por cuenta de ésta, especificando su destino final previsto e importe de la reserva por adquisición de acciones de la sociedad dominante. También se informará sobre el número, valor nominal e importe de la reserva correspondiente a las acciones propias aceptadas en garantía.
\end{enumerate}


\section{Situación fiscal}
(ver variación en caso microempresas)
\begin{enumerate}
 \item Impuestos sobre beneficios.
  \begin{enumerate}
   \item Información relativa a las diferencias temporarias deducibles e imponibles registradas en el balance al cierre del ejercicio.

   \item Antigüedad y plazo previsto de recuperación fiscal de los créditos por bases imponibles negativas.

   \item Incentivos fiscales aplicados en el ejercicio y compromisos asumidos en relación con los mismos.

   \item Provisiones derivadas del impuesto sobre beneficios así como sobre las contingencias de carácter fiscal y sobre acontecimientos posteriores al cierre que supongan una modificación de la normativa fiscal que afecta a los activos y pasivos fiscales registrados. En particular se informará de los ejercicios pendientes de comprobación.

   \item Cualquier otra circunstancia de carácter sustantivo en relación con la situación fiscal.
  \end{enumerate}

\item Otros tributos.
Se informará sobre cualquier circunstancia de carácter significativo en relación con otros tributos, en particular cualquier contingencia de carácter fiscal, así como los ejercicios pendientes de comprobación.
\end{enumerate}


\section{Ingresos y Gastos}
\begin{enumerate}
 \item Se desglosarán las compras y variación de existencias, de mercaderías y de materias primas y otras materias consumibles, dentro de la partida 4. «Aprovisionamientos», del modelo de la cuenta de pérdidas y ganancias. Asimismo, se diferenciarán las compras nacionales, las adquisiciones intracomunitarias y las importaciones.
 Desglose de la partida 7. «Otros gastos de explotación» del modelo de la cuenta de pérdidas y ganancias, especificando el importe de las correcciones valorativas por deterioro de créditos comerciales y los fallidos.

\item El importe de la venta de bienes y prestación de servicios producidos por permuta de bienes no monetarios y servicios.

\item Los resultados originados fuera de la actividad normal de la empresa incluidos en la partida «Otros resultados».
\end{enumerate}

\section{Subvenciones, donaciones y legados}
Se informará sobre:
\begin{enumerate}
 \item El importe y características de las subvenciones, donaciones y legados recibidos que aparecen en el balance, así como los imputados en la cuenta de pérdidas y ganancias.

 \item Análisis del movimiento del contenido de la subagrupación correspondiente del balance, indicando el saldo inicial y final así como los aumentos y disminuciones.

 \item Información sobre el origen de las subvenciones, donaciones y legados, indicando, para las primeras, el Ente público que las concede, precisando si la otorgante de las mismas es la Administración local, autonómica, estatal o internacional.
\end{enumerate}


\section{Operaciones con partes vinculadas}
\begin{enumerate}
 \item La información sobre operaciones con partes vinculadas se suministrará separadamente para cada una de las siguientes categorías:
  \begin{enumerate}
   \item Entidad dominante.
   \item Otras empresas del grupo.
   \item Negocios conjuntos en los que la empresa sea uno de los partícipes.
   \item Empresas asociadas.
   \item Empresas con control conjunto o influencia significativa sobre la empresa.
   \item Personal clave de la dirección de la empresa o de la entidad dominante.
   \item Otras partes vinculadas.
  \end{enumerate}
 \item La empresa facilitará información suficiente para comprender las operaciones con partes vinculadas que haya efectuado y los efectos de las mismas sobre sus estados financieros, incluyendo, entre otros, los siguientes aspectos:
   \begin{enumerate}
     \item Identificación de las personas o empresas con las que se han realizado las operaciones vinculadas, expresando la naturaleza de la relación con cada parte implicada

     \item Detalle de la operación y su cuantificación, informando de los criterios o métodos seguidos para determinar su valor.

     \item Beneficio o pérdida que la operación haya originado en la empresa y descripción de las funciones y riesgos asumidos por cada parte vinculada respecto de la operación.

     \item Importe de los saldos pendientes, tanto activos como pasivos, sus plazos y condiciones, naturaleza de la contraprestación establecida para su liquidación, agrupando los activos y pasivos en los epígrafes que aparecen en el balance de la empresa y garantías otorgadas o recibidas.

     \item Correcciones valorativas por deudas de dudoso cobro o incobrables relacionadas con los saldos pendientes anteriores.
  \end{enumerate}

\item La información anterior podrá presentarse de forma agregada cuando se refiera a partidas de naturaleza similar. En todo caso, se facilitará información de carácter individualizado sobre las operaciones vinculadas que fueran significativas por su cuantía o relevantes para una adecuada comprensión de las cuentas anuales.

\item No será necesario informar en el caso de operaciones que, perteneciendo al tráfico ordinario de la empresa, se efectúen en condiciones normales de mercado, sean de escasa importancia cuantitativa y carezcan de relevancia para expresar la imagen fiel del patrimonio, de la situación financiera y de los resultados de la empresa.

\item No obstante, en todo caso deberá informarse sobre el importe de los sueldos, dietas y remuneraciones de cualquier clase devengados en el curso del ejercicio por el personal de alta dirección y los miembros del órgano de administración, cualquiera que sea su causa, así como del pago de primas de seguros de vida respecto de los miembros antiguos y actuales del órgano de administración y personal de alta dirección. Asimismo, se incluirá información sobre indemnizaciones por cese. Cuando los miembros del órgano de administración sean personas jurídicas, los requerimientos anteriores se referirán a las personas físicas que los representen. Estas informaciones se podrán dar de forma global por concepto retributivo recogiendo separadamente los correspondientes al personal de alta dirección de los relativos a los miembros del órgano de administración.

También deberá informarse sobre el importe de los anticipos y créditos concedidos al personal de alta dirección y a los miembros de los órganos de administración, con indicación del tipo de interés, sus características esenciales y los importes eventualmente devueltos, así como las obligaciones asumidas por cuenta de ellos a título de garantía. Cuando los miembros del órgano de administración sean personas jurídicas, los requerimientos anteriores se referirán a las personas físicas que los representen. Estas informaciones se podrán dar de forma global por cada categoría, recogiendo separadamente los correspondientes al personal de alta dirección de los relativos a los miembros del órgano de administración.

\item Las empresas que se organicen bajo la forma jurídica de sociedad anónima, deberán especificar la participación de los administradores en el capital de otra sociedad con el mismo, análogo o complementario género de actividad al que constituya el objeto social, así como los cargos o las funciones que en ella ejerzan, así como la realización por cuenta propia o ajena, del mismo, análogo o complementario género de actividad del que constituya el objeto social de la empresa.
\end{enumerate}

\section{Otra información}
Se incluirá información sobre:
\begin{enumerate}
 \item El número medio de personas empleadas en el curso del ejercicio, expresado por categorías.
 \item La naturaleza y el propósito de negocio de los acuerdos de la empresa que no figuren en balance y sobre los que no se haya incorporado información en otra nota de la memoria, así como su posible impacto financiero, siempre que esta información sea significativa y de ayuda para la determinación de la posición financiera de la empresa.
\end{enumerate}

% FIN_CUERPO
\end{document}
